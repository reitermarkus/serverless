\chapter{Results}
\label{sec:results}

\section{Benchmarks}

For our test setup, we deployed our stack on an \textit{Intel NUC}, which is a fairly inexpensive
mini PC. With the following benchmarks, we want to show what such an affordable machine is capable
of when coupled with other cheap IoT devices, e.g. a Raspberry Pi.

\begin{table}[H]
  \centering
  \begin{tabular}{|l|l|l|r|}
    \hline
    Name                     & Model              & Processor                                                                             & Memory              \\ \hline
    \whitelist{Intel NUC}    & \whitelist{8I5BEH} & \whitelist{Intel Core i5-8259U} @ \SI{2.30}{\giga\hertz}                              & \SI{16}{\giga\byte} \\ \hline
    \whitelist{Raspberry Pi} & 3 B+               & \whitelist{Broadcom BCM2837B0, 64-bit SoC} @ \SI{1.4}{\giga\hertz}                    & \SI{1}{\giga\byte}  \\ \hline
    Desktop PC               & –                  & \whitelist{Intel Core i9-7900X} @ \SI{4.7}{\giga\hertz}                               & \SI{64}{\giga\byte} \\ \hline
  \end{tabular}
  \caption{Hardware used for Benchmark}
  \label{tab:benchmark-hardware}
\end{table}

The specific hardware we used for benchmarking is listed in \autoref{tab:benchmark-hardware} above.
The desktop PC and the \textit{NUC} were both connected via \whitelist{ethernet} to a \SI{1}{\giga\bit} network
switch. The \textit{Raspberry Pi} was connected via WiFi to an access point connected to the same
switch. \autoref{tab:benchmark-network-speed} shows the link speed between the hardware components
as tested with \texttt{iperf3} \cite{iperf}.

\begin{table}[H]
  \centering
  \begin{tabular}{|l|r|}
    \hline
    Connection                                  & Speed                          \\ \hline
    \textit{NUC} $\leftrightarrow$ Desktop PC   & \SI{940}{\mega\bit\per\second} \\ \hline
    \textit{NUC} $\leftrightarrow$ Raspberry Pi & \SI{55}{\mega\bit\per\second}  \\ \hline
  \end{tabular}
  \caption{Link speed between Hardware Components}
  \label{tab:benchmark-network-speed}
\end{table}


\begin{table}[H]
  \centering
  \begin{tabular}{|r|r|}
    \hline
    Number of Sensors & Messages per Second \\ \hline
                    1 &                7.55 \\ \hline
                    8 &               49.20 \\ \hline
  \end{tabular}
  \caption{Benchmark for Messages per Second from a Raspberry Pi to a \whitelist{NUC}}
  \label{tab:benchmark-nuc-raspberry-pi}
\end{table}

In \autoref{tab:benchmark-nuc-raspberry-pi}, we see the benchmark data for the \texttt{log-data}
function. For this benchmark, we set the \textit{Raspberry Pi} to not use any delay between
measurements instead of the usual measurement interval of \SI{15}{\second}. We also disabled the
\texttt{CPU Load Aggregate} measurement, since this has an inherent delay of \SI{1}{\second} due to
the fact that it calculates the load average for a given duration. The \textit{Raspberry Pi}
application collects the data for all sensors in each iteration, which means that we can actually
get much more throughput with eight sensors than with a single sensor. Thus, we can further
conclude that the time to produce a measurement is negligible compared to the network latency.

In order to test this theory, we decided to use parallel \texttt{curl} requests from a more
powerful computer to test the limit of the \textit{NUC}.

\begin{table}[H]
  \centering
  \begin{tabular}{|r|r|}
    \hline
    Number of concurrent Requests & Messages per Second \\ \hline
                                1 &                61.2 \\ \hline
                               10 &               239.0 \\ \hline
                               20 &               322.4 \\ \hline
                               50 &               379.5 \\ \hline
  \end{tabular}
  \caption{Benchmark for Messages per Second from a Desktop PC to a \whitelist{NUC}}
  \label{tab:benchmark-nuc-curl}
\end{table}

Each \texttt{curl} request in \autoref{tab:benchmark-nuc-curl} contains a single sensor record which
corresponds to the case in \autoref{tab:benchmark-nuc-raspberry-pi} with a single sensor. By looking
at this data, we can see when only sending data for a single sensor, the \textit{Raspberry Pi} is
clearly the bottleneck, not the \textit{NUC}. It is safe to assume that the \textit{NUC} is able to
handle approximately 400 requests per second.

With this assumption that a \textit{NUC} can handle 400 requests per second, we can deduce that our
\textit{NUC} is able to process streams from between

$\frac{400}{7.55} \times 1 \approx 53$

and

$\frac{400}{49.20} \times 8 \approx 65$

sensors.

This being a synthetic benchmark, it is likely that in a real world scenario, where sensor data is
sent only every \textit{x} seconds, the number of supported sensors is actually in the 100s instead
of only in the 10s.
