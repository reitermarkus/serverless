\chapter{Architecture}
\label{sec:architecture}

\begin{figure}[H]
  \adjincludegraphics[max width=\textwidth]{architecture-diagram}
  \caption{Diagram showing how the parts of our architecture work together.
    \nocite{smartphone-icon, browser-icon}
  }
  \label{fig:architecture-diagram}
\end{figure}

Our task is to do IoT data analytics using serverless computing, therefore, as a first step, we
started out by thinking about the underlying infrastructure. We decided on using \textit{Apache
Kafka} for streaming, \textit{MongoDB} as our database, \textit{OpenFaaS} as the serverless platform
and \textit{Rust} as the programming language for our serverless functions. The whole stack can then
be initialised via a \textit{Ruby Rake} file, which acts as our main deployment script. In
\autoref{fig:architecture-diagram}, you can see how all of the parts in our architecture work
together.

\section{How does it work?}

Let us start with one of the edge devices in our architecture, a smartphone. A smartphone (bottom
left of \autoref{fig:architecture-diagram}) collects data from its internal sensors (e.g.
temperature, gravity, orientation) and then sends it to a \textit{Kafka} message broker in order to
be processed further. For this, we implemented a cross platform mobile application. A
\textit{Raspberry Pi} (also bottom left of \autoref{fig:architecture-diagram}) collects data from
external sensors (e.g. air humidity, air pressure, luminosity) as well as internal sensors (e.g. CPU
temperature, CPU load). Similarly to the smartphone, it sends the collected sensor data to a
\textit{Kafka} message broker. For the \textit{Raspberry Pi}, we developed an application written in
Rust.

The \textit{Kafka} message broker (centre left in \autoref{fig:architecture-diagram}) then forwards
this stream of sensor messages to the serverless stack (\textit{OpenFaaS} in top right corner of
\autoref{fig:architecture-diagram}). Each sensor message is then processed by a log function in the
serverless stack (\texttt{log-data}) which in turn calls the database function (\texttt{database}).
The data is then persisted in a \textit{MongoDB} database (centre of
\autoref{fig:architecture-diagram}).

The last step is to visualise the collected data, which is done with a web interface provided by the
\texttt{ui} function. This interface can then be accessed by any web browser, as shown in the bottom
right corner of \autoref{fig:architecture-diagram}. The interface is also accessible from within
the mobile application.

\section{Hardware}

Note how in \autoref{fig:architecture-diagram}, the cloud part is nested inside the fog device part:
In our test setup, our serverless stack is actually running on a fog device instead of in the cloud,
which already highlights the flexibility of serverless computing. The fog device in our case is an
\textit{Intel NUC} (\autoref{fig:intel-nuc}), a mini PC.

\begin{figure}[H]
  \centering
  \adjincludegraphics[height=10em]{intel-nuc}
  \caption{\whitelist{Intel NUC}}
  \label{fig:intel-nuc}
\end{figure}

A \textit{Raspberry Pi}, arguably one of the most popular SoC (System-on-a-Chip) computers,
functions as our edge device for collecting data from external sensors. Specifically, we are using a
\textit{Raspberry Pi} 3, as seen in \autoref{fig:raspberry-pi-3}.

\begin{figure}[H]
  \centering
  \adjincludegraphics[height=14em]{raspberry-pi-3}
  \caption{Raspberry Pi 3}
  \label{fig:raspberry-pi-3}
\end{figure}

The mobile application works on any smartphone running Android or iOS.
