\section{Raspberry Pi}

In order for us to gather data, we decided on using multiple Raspberry Pis with different sensors
attached to them. We chose the Raspberry Pi because it is backed by a huge community and the vast
amount of documentation available for it. Having documentation available was essential since we
wanted to write the client application for the Raspberry Pi in Rust. This way we could validate our
Rust code by comparing it to example code written in other programming languages, most commonly
Python or C in the case of the Raspberry Pi.

\subsection{Rust on the Raspberry Pi}

For our first “Hello, world!” program which would run on the Raspberry Pi, we took the simplest
approach at the time. We would write the code on our development machines and synchronize the code
to the Raspberry Pi using \texttt{rsync} and the compile and run it via \texttt{ssh}. This worked
fine at the time. Once we got to the point where we needed to add more dependencies for the various
sensors and networking, compile times naturally increased to the point at which it simply wasn't
feasible anymore to compile directly using the inadequate processor of the Raspberry Pi. A single
build was approaching a compile time of around five minutes, so we had to start looking for
alternatives.

\subsection{Cross Compilation}

Soon after we realised that compiling directly on the Raspberry Pi was not a good solution, we had
to find a way to cross compile for the Raspberry Pi. This was further complicated by the fact the we
were using \textit{macOS} and \textit{Windows}, so none of the pre-compiled cross compilation
toolchains for Linux were an option for us.

We then found the \texttt{cross} tool, which claimed to automatically install the corresponding
toolchain and then run the cross compilation in a Docker container, so this should have worked on
both \textit{macOS} and \textit{Windows}. Right after we installed it using \texttt{cargo install
cross}, we saw that \textit{macOS} and \textit{Windows} support was lacking. On both platforms,
tried to simply call \texttt{cargo} directly instead of running in a Docker container. Since this
tool still seemed to be the best solution we could find for our situation and since the tool is
open-source, we decided to dig into the source code and add the missing \textit{macOS} and
\textit{Windows} support ourselves.

Getting \texttt{cross} to actually run a Docker container on both of our platforms was quite easy,
we only had to add two new definitions, one for \textit{macOS} (\texttt{x86\_64-apple-darwin}) and
one for \textit{Windows} (\texttt{x86\_64-pc-windows-msvc}). The next problem we were facing was
that inside of the Docker container, the \texttt{CARGO\_HOME} directory was mounted, and with it,
its \texttt{bin} subdirectory. This meant that the Docker container would first look in this
directory instead of the respective toolchain root for the corresponding target's binaries. Since
the binaries in \texttt{CARGO\_HOME/bin} are compiled for the host machine, these previously worked
fine since only \texttt{x86\_64} \textit{Linux} hosts were supported. Now however, it would try
running a \textit{macOS} or \textit{Windows} binary inside of a \textit{Linux} Docker container.
This was not as straight-forward to debug as it might seem, since the error message looked like a
syntax error in a shell command. Once we found what the problem was, the next challenge was to
actually implement the fix for it. We still had to mount \texttt{CARGO\_HOME}, but exclude its
\texttt{bin} subdirectory. Since the contents of \texttt{CARGO\_HOME} can vary depending on what you
have installed, we could not simply mount each subdirectory individually and exclude \texttt{bin},
so the only way was to use a somewhat hacky workaround.

The whole \texttt{CARGO\_HOME} directory was already mounted using

\begin{lstlisting}[language=Bash]
  -v "$CARGO_HOME:/cargo:Z"
\end{lstlisting}

In order to prevent mounting the \texttt{bin} subdirectory from the host machine, we added

\begin{lstlisting}[language=Bash]
  -v /cargo/bin
\end{lstlisting}

This means that technically the \texttt{bin} subdirectory is still mounted, but is overlaid by
another virtual volume which is not bound to a directory on the host, which effectively prevents
non-Linux binaries being accessible inside the Docker container.
