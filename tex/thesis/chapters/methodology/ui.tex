\subsection{Serverless UI}

Another instrumental part of the project is the UI. Without any form of visualization our whole data
gathering process would not be of much use. For this reason we decided early on what kind of front
end web development stack we would use. Since both us are familiar with \textit{React} - a
\textit{JavaScript} famous for revolutionizing the use of \textit{virtual DOM}
to render \textit{UI} elements and to update \textit{UI} elements reactively on
UI change - we decided on using \textit{MarkoJS}.

\subsubsection{MarkoJS}

\textit{MarkoJS} shares many of the same benefits as \textit{React} with some added flexibility like
\textit{concise HTML syntax} which makes the whole \textit{markup} more readable and easier to
write. It also has the ability to use conventional control flow structures like \textit{if} or
\textit{for} directly in the \textit{markup}.

\subsubsection{Babel}

However \textit{MarkoJS} is most and foremost a \textit{JavaScript} framework and due to the nature
of the before mentioned features, \textit{transpiling} is inevitable. \textit{Transpiling} means to
transform modern possibly unsupported \textit{JavaScript} into valid \textit{ECMAScript 5} compliant
code that any browser can understand. For this process we currently use the industry standard
technology \textit{Babel}. With \textit{Babel} \textit{transpiling} is rather easy. All necessary
definitions are in a \textit{webpack.config.babel.js} file which brings us to the next essential
tool \textit{Webpack}.

\subsubsection{Webpack}

“Webpack is a module bundler. Its main purpose is to bundle JavaScript files for usage in a browser,
yet it is also capable of transforming, bundling, or packaging just about any resource or asset.”

With that being said \textit{Webpack} can to some extend be considered as the main part of the whole
\textit{front end stack} as it is responsible for orchestrating \textit{transpiling} of
\textit{MarkoJS} files. It is also responsible for providing \textit{Webpack Dev Server}, a server
that reloads on file change. Furthermore \textit{Webpack} runs all files through certain
optimization plugins on release build, which can bring down the size of the \textit{code bundle}
quite considerably. \textit{Webpack} is also able to transform \textit{SASS} into \textit{CSS}.

\subsubsection{SASS}

\textit{SASS} is a superset of \textit{CSS} with many additional features like variables, functions,
nesting and exporting / importing files. We make heavy use of those features to structure and
modularise our \textit{CSS} code.

\subsection{UI}

The \textit{UI} itself uses a basic layout where all sensor devices registered in the database are
listed on the side in a so called \textit{sidebar}. The user then can click on each individual
device to open a view  with detailed graphs of sensor data of that device.