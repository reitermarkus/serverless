\section{Background}

In this part we try to make the reader more familiar with the individual technologies and their
respective terms, that will be used throughout the thesis. Firstly we start by discussing the main
topic: Serverless Computing, we then will go on in more detail about \textit{Docker} and our
specific use case.

\subsection{Serverless Computing}

Serverless computing is a new emerging paradigm for deploying applications into the cloud. It gained
in popularity in recent years largely due to shift of enterprise application deployment in
containers and microservices. Serverless Computing offers developers a simplified programming model
for creating cloud applications while minimising operational concerns. \cite{servprog}

The term “serverless” can be confusing, since physical server hardware is of course still needed in
order to run applications. The main point is that the application user or developer does not need to
manage scaling, plan for variable capacity or maintain any other aspect of the servers. All of this
is provided as a service from the cloud provider. \cite{wikiservcomp}

\subsubsection{Layers of Cloud Computing}

\begin{figure}[H]
  \centering
  \adjincludegraphics[max width=\textwidth]{cloud-history}
  \caption{History of cloud computing: going from Data Centre to IaaS, to PaaS, to finally
  Serverless (FaaS) \cite{layercloudcomp}}
\end{figure}

Historically, each new paradigm in the space of cloud computing has brought with it a new layer of
abstraction. First, there was the move from managing physical hardware in a data centre to being
able to rent infrastructure from a cloud provider. This layer, called IaaS (Infrastructure as a
Service), shifted the need for the customer to manage hardware infrastructure to the cloud
provider. This change also improved scalability as customers could now rent infrastructure on a
pay-as-you-go basis instead of paying for servers which would be idle most of the time.

With IaaS, the customer is still responsible for managing the setup of the rented infrastructure,
i.e. installing the necessary dependencies needed to run a given application. Naturally, the next
layer of abstraction is to provide the customer with an environment suited to run an application
without the need to manually install a programming language and any dependencies. This layer is
called PaaS (Platform as a Service). Using this layer of abstraction, the user does have to worry
neither about managing the underlying hardware nor about managing the operating system the
application is running on.

Now, in the era of serverless computing, there is yet again a new layer of abstraction. Called
FaaS (Function as a Service), this layer provides a runtime environment for a given language in
order to run functions. An application deployed in a FaaS environment consists of multiple
functions interacting with one another whereas in a PaaS environment, an application is deployed
as a single unit.

\subsection{Docker}

Docker is a technology which is used to run software packaged into containers. Containers are
self-contained and provide an isolated environment for software to run in. A container includes
all configuration files and dependencies as well as the software itself, which makes it highly
portable. In the case of \textit{Docker}, these stand-alone container images can then be executed
by the \textit{Docker Engine}, which turns the stored images into running containers. For this
reason, every container image contains a customisable entry point command which is executed when
the container is started.

\begin{figure}[H]
  \centering
  \adjincludegraphics[max width=\textwidth]{docker-containerized-and-vm}
  \caption{Comparison of Containerised Applications and Virtual Machines \cite{docker-container}}
\end{figure}

\textit{Docker} containers are standardised, which means they can run virtually anywhere the
\textit{Docker Engine} is supported, be it on Linux, Windows, in data centres or the cloud.
Compared to virtual machines, containers are very lightweight since they share the host machine's
system kernel and thus don't require a separate operating system for each application. Despite
not using a separate operating, containers are completely isolated from the host system as well as
other containers by default.

Furthermore, since containers usually don't contain a full operating system, startup times are
also much faster compared to virtual machines, which have to completely boot the entire operating
system from scratch in addition to starting the application. \cite{docker-container}

\subsection{OpenFaaS}

"OpenFaaS - Serverless Functions Made Simple”. As the slogan already entails, \textit{OpenFaaS} is a
framework for the deployment of serverless functions. In contrast to competing hosted products like
\textit{AWS Lambda} or \textit{Azure Cloud Functions} it offers a lot more flexibility in the way
functions are written and hardware resources are managed. In fact any programming language that one
can think of can be used to write functions for \textit{OpenFaaS} as long as they are provided with
a \textit{Dockerfile}. Deploying \textit{OpenFaaS's} stack in a cluster is easy as well, as the
framework poses itself to have first class support for \textit{Docker Swarm} and being
\textit{Kubernetes} native. \cite{openfaas-docs}

\subsection{MongoDB}

\subsection{Apache ZooKeeper}

ZooKeeper is a service for centrally managing configuration, naming, handling distributed
synchronisation and providing group services. Many distributed applications need access to
some or all of these kinds of services, so ZooKeeper is a great way to avoid reinventing the
wheel. This is especially true given the fact that these highly distributed scenarios bring with
them a vast amount of potential bugs and race conditions every time these services would have to
be implemented from scratch. It is incredibly hard to get these things completely right the first
time, and by using ZooKeeper all of these potential problems are nullified and a lot of time can
be saved which can be spent developing the actual application logic rather than spending it on
debugging synchronisation problems or managing configuration across a distributed network.
Particularly in the long run, using ZooKeeper helps reduce unforeseen complexity for applications
continuously increasing in size. \cite{zookeeper-homepage}

\subsection{Apache Kafka}
