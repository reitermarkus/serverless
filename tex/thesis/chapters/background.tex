\section{Background}

In this part we try to make the reader more familiar with the individual technologies and their
respective terms that will be used throughout the thesis. Firstly we start by discussing the main
topic, \textit{Serverless Computing}, we will then go on in more detail about \textit{Docker} and
our specific use case.

\subsection{Serverless Computing}

Serverless computing is a new emerging paradigm for deploying applications into the cloud. It gained
in popularity in recent years largely due to shift of enterprise application deployment in
containers and microservices. \textit{Serverless Computing} offers developers a simplified
programming model for creating cloud applications while minimising operational concerns.
\cite{servprog}

The term “serverless” can be confusing, since physical server hardware is of course still needed in
order to run applications. The main point is that the application user or developer does not need to
manage scaling, plan for variable capacity or maintain any other aspect of the servers. The
management of all of these things is provided as a service from the cloud provider.
\cite{wikiservcomp}

\subsubsection{Layers of Cloud Computing}

\begin{figure}[H]
  \centering
  \adjincludegraphics[max width=\textwidth]{cloud-history}
  \caption{History of cloud computing: going from Data Centre to IaaS, to PaaS, to finally
  Serverless (FaaS) \cite{layercloudcomp}}
\end{figure}

Historically, each new paradigm in the space of cloud computing has brought with it a new layer of
abstraction. First, there was the move from managing physical hardware in a data centre to being
able to rent infrastructure from a cloud provider. This layer, called IaaS (Infrastructure as a
Service), shifted the need for the customer to manage hardware infrastructure onto the cloud
provider. This change also improved scalability as customers could now rent infrastructure based on
the pay-as-you-go principle instead of paying for large servers which would be idle most of the
time.

With IaaS, the customer is still responsible for managing the setup of the rented infrastructure,
i.e. installing the necessary dependencies needed to run a given application. Naturally, the next
layer of abstraction is to provide the customer with an environment suited to run an application
without the need to manually install a programming language or any dependencies. This layer is
called PaaS (Platform as a Service). Using this layer of abstraction, the user does have to worry
neither about managing the underlying hardware nor about managing the operating system the
application is running on.

Now, in the era of serverless computing, there is yet again a new layer of abstraction. Called FaaS
(Function as a Service), this layer provides a runtime environment for a given language in which to
run individual functions in. An application deployed in a FaaS environment consists of multiple
functions interacting with one another whereas in a PaaS environment, an application is deployed as
a single unit.

\subsection{Docker}

Docker is a technology which is used to run software packaged into containers. Containers are
self-contained and provide an isolated environment for software to run in. A container includes all
configuration files and dependencies as well as the software itself, which makes it highly portable.
In the case of \textit{Docker}, these stand-alone container images can then be executed by the
\textit{Docker Engine}, which turns the stored images into running containers. For this reason,
every container image contains a customisable \textit{entry point} command which is executed when
the container is started.

\begin{figure}[H]
  \centering
  \adjincludegraphics[max width=\textwidth]{docker-containerized-and-vm}
  \caption{Comparison of Containerised Applications and Virtual Machines \cite{docker-container}}
\end{figure}

\textit{Docker} containers are standardised, which means they can run virtually anywhere the
\textit{Docker Engine} is supported, be it on Linux, Windows, in data centres or in the cloud.
Compared to virtual machines, containers are very lightweight since they share the host machine's
system kernel and thus don't require a separate operating system for each application. Despite not
using a separate operating system, containers are completely isolated from the host system as well
as other containers by default.

Furthermore, since containers usually don't contain a full operating system, startup times are also
much faster compared to virtual machines, which have to completely boot the entire operating system
from scratch in addition to starting the application. \cite{docker-container}

\subsection{OpenFaaS}

“OpenFaaS – Serverless Functions Made Simple”. As the slogan already entails, \textit{OpenFaaS} is a
framework for the deployment of serverless functions. In contrast to competing hosted products like
\textit{AWS Lambda} or \textit{Azure Cloud Functions}, it offers a lot more flexibility regarding
the way functions are written and hardware resources are managed. In fact, any programming language
imaginable can be used to write functions for \textit{OpenFaaS}, as long as that language can run or
be compiled in a \textit{Docker} container. Deploying an \textit{OpenFaaS} stack in a cluster is
easy as well, as the framework poses itself for having first class support for \textit{Docker Swarm}
and being \textit{Kubernetes} native. \cite{openfaas-docs}

\begin{figure}[H]
  \centering
  \adjincludegraphics[max width=\textwidth]{of-workflow}
  \caption{\textit{OpenFaaS} workflow \cite{openfaas-docs}}
\end{figure}

\subsection{MongoDB}

\textit{MongoDB} is a document database that is scalable and flexible. The data in a
 \textit{MongoDB} database is stored in flexible, \textit{JSON}-like documents, making it easy to
 work with. The document database offers powerful ways to access and analyse data with features like
 ad hoc querying and real time aggregation. Since \textit{MongoDB} is a distributed database at its
 core, high availability and horizontal scaling are already built in. Furthermore, due to its
 “real-time” nature, \textit{MongoDB} lends itself very well for \textit{IoT} applications.
 \cite{mongodb-description}

 \begin{figure}[H]
  \centering
  \begin{lstlisting}[basicstyle=\small]
{
  _id: '5cf0029caff5056591b0ce7d',
  name: 'Jane Wu',
  address: {
    street: '1 Circle Rd',
    city: 'Los Angeles',
  }
}
  \end{lstlisting}
  \caption{A \textit{MongoDB} document (adapted from \cite{mongodb-description}).}
\end{figure}

\subsection{Apache \whitelist{ZooKeeper}}
\label{sec:background-zookeeper}

\textit{ZooKeeper} is a service for centrally managing configuration, naming, handling distributed
synchronisation and providing group services. Many distributed applications need access to some or
all of these kinds of services, so \textit{ZooKeeper} is a great way to avoid reinventing the wheel
in that respect. This is especially true given the fact that these highly distributed scenarios
bring with them a vast amount of potential bugs and race conditions every time these services would
have to be implemented from scratch. It is incredibly hard to get these things completely right the
first time and by using \textit{ZooKeeper}, all of these potential problems are nullified and a lot
of time can be saved which can consequently be spent developing the actual application logic rather
than spending it on debugging synchronisation problems or managing configuration across a
distributed network. Particularly in the long run, using \textit{ZooKeeper} helps reduce unforeseen
complexity for applications continuously increasing in size.
\cite{zookeeper-homepage}

\subsection{Apache Kafka}
\label{sec:background-kafka}

\textit{Apache Kafka} is a distributed streaming platform. The first of its three key capabilities
is the ability of publishing and subscribing to streams of records, akin to message queues.
Secondly, the fault-tolerant storage of these streams of records over a long period of time is
essential. The final piece of the puzzle is the real-time processing of these streams.

\textit{Kafka} is usually used to create real-time pipelines for passing data between distributed
applications or to transform streams of data in real-time. \textit{Kafka} is deployed as a cluster
on a single machine or on multiple servers and groups streams into categories called “topics”. Each
record consists of a key-value pair and a timestamp to provide the possibility to process streams
chronologically.

\begin{figure}[H]
  \centering
  \adjincludegraphics[max width=\textwidth]{kafka-apis}
  \caption{Visualisation of \textit{Kafka} APIs \cite{kafka-complete-introduction}}
  \label{fig:kafka-apis}
\end{figure}

As seen in \autoref{fig:kafka-apis}, there are four core APIs provided by \textit{Kafka}. The
Producer and Consumer APIs are used to either post records to a topic or to receive records from a
topic, respectively. Then there is the Streams API, which is used to essentially tie an application
in between one or multiple input streams and one or multiple output streams. Lastly, there is the
Connector API, which allows linking \textit{Kafka} to external systems, this is essential for
connecting serverless functions to streams.
\cite{kafka-introduction}

\subsection{Rust}

\textit{Rust} is a new strongly typed open source programming language supported by
\textit{Mozilla}. The basic idea of the language is to offer low level system access with focus on
security, performance and concurrency. \cite{rustbook1, forkjoin}

In a way, \textit{Rust} is an alternative to \textit{C}/\textit{C++} with the difference of it
having support for high level programming language concepts like pattern matching, closures and safe
memory management. \cite{rustbook1, forkjoin}

Memory management actually differs quite considerably compared to conventional languages like
\textit{Java} or \textit{C}/\textit{C++}. \textit{Rust} does not use a garbage collector, in fact,
thanks its ownership model, memory does not need to be freed manually at all, avoiding common
pitfalls like a double-free error.

Like many other modern programming languages, \textit{Rust} has its own package manager, called
\textit{Cargo}. To some extent, \textit{Cargo} can be compared with \textit{NPM} from the
\textit{JavaScript} ecosystem. However the amount of packages (a.k.a. \textit{Crates}) available in
\textit{Rust} is substantially lower compared to \textit{NPM} as \textit{Rust} is a much younger
language with a smaller community.

Due to \textit{Rust}'s strong emphasis on ownership and typing, many concurrency issues are already
caught at compile time. Data races are prevented as it is impossible to have multiple shared
references to a variable without synchronisation. This concept is also referred as “fearless
concurrency”. \cite{rustbook2}

\subsection{Flutter}

\textit{Flutter} is a \textit{UI} toolkit from \textit{Google} for building natively compiled mobile
applications. With the same code base, the application can also be deployed on web and desktop.
\textit{Flutter} offers many features that make the development experience easier, like
\textit{Stateful Hot Reload} and filtering application specific error messages on \textit{Android}
by default. Although \textit{Flutter} is a cross platform framework, it is said to still rival
native performance on \textit{Android} and \textit{iOS} by directly transforming \textit{Flutter}
code into native code.
\cite{flutter}

\textit{Flutter} uses \textit{Dart} as its programming language. So the aforementioned \textit{Flutter}
code is actually \textit{Dart} code. \textit{Dart} itself has many language extensions that are well
suited for mobile application development. For example, the language offers a mature and complete
\textit{async-await} implementation with isolate-based concurrency support. Furthermore its syntax
is familiar and seems inspired by other popular modern languages. As mentioned before, \textit{Dart}
code is transpiled to the platform's native instruction set, this means native \textit{ARM} code on
mobile devices and \textit{JavaScript} on the web. \cite{dart}

\subsection{Azure Pipelines}

\textit{Azure Pipelines} is a CI (continuous integration) service provided by Microsoft which is
used to automatically build and test projects. Given that all major platforms, i.e. Linux, macOS and
Windows, are supported, projects written in virtually any language can be built and tested. In
addition to continuous integration, \textit{Azure Pipelines} also offers CD (continuous delivery),
which can be used to directly publish any build artefacts, e.g. compiled binaries. Like many other
CI providers, \textit{Azure Pipelines} provides a free plan for open-source projects.
\cite{azure-pipelines}

Continuous integration is a very important measure for ensuring that new changes to a project don't
break existing functionality. Without CI testing, those breaking changes would result in much more
work when they come to light in the future.
