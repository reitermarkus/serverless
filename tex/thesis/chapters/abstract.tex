\section*{Abstract}

In the grand scheme of things, \textit{Cloud Computing} is still a relative new development which
has seen rapid growth in recent years and with that has seen a lot of technologies emerge around it.
One of those is the concept of \textit{Serverless Computing} or \textit{Function as a Service
(FaaS)}. Due to its versatile nature, a lot of big players in the cloud computing field jumped on it
and have their own implementation in their portfolio.

Likewise, the \textit{Internet of Things (IoT)} has also seen large adoption in many aspects of life
and is an integral part of it. Be it for household appliances or in industrial applications,
\textit{IoT} devices can be found everywhere.

This bachelor thesis tries to show the usefulness of \textit{Serverless Computing} in conjunction
with \textit{IoT} devices by building an infrastructure to host \textit{serverless functions} where
all \textit{IoT} devices can then send their data to. While not a classic \textit{IoT} device, there
is also support for both \textit{Android} and \textit{iOS} devices to transmit data. This data is
then further used for analytics and visualised with graphs in a web interface. In order to achieve
this we decided to use the \textit{OpenFaaS} framework to realise the hosting function platform.
Furthermore, \textit{Kafka} is used to manage all incoming requests from devices. \textit{Kafka}
also handles the forwarding of requests to functions. For persistence of sensor data we are using
the \textit{NoSQL} database \textit{MongoDB}. The database is of great help when confronted with
data of \textit{IoT} devices. Having a stack like this also means dealing with a lot of
configuration. To aid this process, we also developed a \textit{Rake} script which makes everything
runnable with only one command. And finally, we are using \textit{Azure Pipelines} as our
verification tool to test all parts of our stack and push our \textit{serverless functions} to an
online registry, which makes deployment even faster.

In the final chapter of this thesis, we present results including benchmarks for running our stack
on affordable hardware, showing just how cost-effective and yet powerful a serverless stack can be.
