\chapter{Motivation}
\label{sec:motivation}

In this section, we describe our motivation for exploring IoT data analysis using serverless
computing.

\section{Flexibility}

Using a serverless approach for building software inherently provides more flexibility than the
traditional approach of using a monolithic architecture. This applies to both the development
process itself as well as the much more diverse deployment options presented by serverless
computing. Traditionally, you would have to choose a single programming language for a project since
using multiple would increase complexity of both the development phase as well as the deployment
strategy. Also, the initial setup of a single server and its ongoing maintenance for use with only a
single language is much easier than having to maintain a server running multiple different
languages. In this regard, serverless computing provides much greater flexibility.

\section{Modularity}

The traditional approach of a monolithic architecture is by nature error prone and not failure
tolerant. If one component crashes, all other components on that server might go down as well. With
serverless functions, this problem can be avoided as every function is an individual item and
therefore isolated from other functions. This fact makes maintaining systems much easier as problems
are faster and more efficient to identify. Testing is also simpler as tests can be written for
individual functions instead of for chunks in a larger code base.

\section{Efficiency}

Traditional software stacks often require considerable hardware in order to run software
efficiently. In our thesis we try to show that rich functionality on the server side can also be
achieved without expensive hardware with the power of serverless computing. Since the heavy lifting
is done by the serverless stack and the clients in our case only have to read sensor data and invoke
web requests, almost any cheap micro-controller is sufficient for processing.

\section{Portability}

Another one of the advantages of serverless computing is portability as one can easily transfer all
functions from one provider to another. In our case however, we try to take things further as not
only the functions are portable, the whole stack in a sense is portable. Every major component of
the stack is composed of containers, which means as long as \textit{Docker} is installed on the host
machine, the stack should be able to run on it.
