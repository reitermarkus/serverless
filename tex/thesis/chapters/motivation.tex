\section{Motivation}

In this section, we describe our motivation for exploring IoT data analysis using serverless
computing.


\subsection{Flexibility}

Using a serverless approach for building software inherently provides more flexibility than the
traditional approach using a monolithic architecture. This applies to both the development process
itself as well as the much more diverse deployment options. Traditionally, you would have to choose
a single programming language for a project since using multiple would increase complexity of both
the development phase as well as the deployment strategy. Also, the initial setup of a single server
and its ongoing maintenance cost using only one language is much easier than having to maintain a
server running multiple different languages. In this regard, serverless computing provides much
greater flexibility.

A program is built from separate functions interacting with one another. These
functions usually communicate via a JSON API. Functions are programming language agnostic, so as
long as the language you are using can invoke a web request to call another function, you can use it
to develop a function. This gives the developer the freedom to choose the best language for a given
task, rather than sticking to a single language which might not be ideal for some specific tasks.
Additionally, deployment of functions is managed by the severless stack, so there is no maintenance
overhead after the serverless stack is deployed.

Every function is a self-contained unit of execution, usually in the form of a container which can
be deployed on the serverless stack. This means that the programming language runtime environment
and all dependencies are therefore also contained in that same container. This allows the developer
to not only use different programming languages but also different versions of the same language
without any maintenance difficulties compared to doing the same on a single server.

\subsection{Modularity}

The traditional approach of a monolithic architecture is by nature error prone and not failure
tolerant. If one component crashes all other components on that server might go down as well. With
serverless functions this problem can be avoided as every function is a individual item and
therefore isolated from other functions. This fact makes maintaining systems much easier as problems
are faster and more efficiently identifiable. Testing is also simpler as one can write tests per
function instead of chunks in a larger code segment.

\subsection{Efficiency}

Traditional software stacks often require considerable hardware in order to run software
efficiently. In our thesis we try to show that rich functionality on the server side can also be
achieved without expensive hardware with the power of serverless computing. Since the heavy lifting
is done by the serverless stack and the clients in our case only have to read sensor data, invoke
web requests, almost any cheap micro-controller is sufficient for processing.

\subsection{Portability}

One of the advantages of serverless computing is portability as one can easily transfer all
functions from one provider to another provider. In our case however we try to take things further as
not only the functions are portable, the whole stack in a sense is portable. Every major component
of the stack is composed of containers, which means as long as \textit{Docker} is installed on the
host machine the stack should be able to install.
