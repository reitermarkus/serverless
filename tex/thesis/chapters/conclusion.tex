\section{Conclusion and Future Work}

As in the thesis explained, building a infrastructure like the one we have, is not a trivial task
and requires an extensive amount of research. The final result then however shows the flexibility
and benefits of our stack and by extension of \textit{Serverless Computing}. It is trivial to switch
the programming environment as one only has to write a new function and provide a corresponding
\textit{Dockerfile}. In terms of \textit{IoT} devices, \textit{FaaS} seems to be a fitting concept
as well, since it offers high scalability and parallelism of invocations, which means the stack can
support an arbitrary amount of devices and is mostly only limited by the hardware it is running on.
While \textit{Kafka} adds a bit of overhead, its managing capabilities are especially useful in the
case of a high amount of connected clients. The choice of a host for persisting data in a project
like this is also very important. \textit{MongoDB} proves to be capable in that regard. With its
\textit{JSON} inspired documents, it makes processing easy for devices with low hardware
specifications as representing data with \textit{JSON} is straight forward. Unfortunately the same
cannot be said, when dealing with mobile applications in a \textit{cross platform} way. The
\textit{cross platform} aspect in our project basically limits itself to the user interface. Newer
versions of both \textit{Android} and \textit{iOS} do not work well when confronted with the
requirement of always having to send data in the background, as both platforms impose energy saving
restrictions to a degree and therefore prevent the application from working as intended.

Ultimately is our project more supposed to act as a proof of concept and give an insight what could
be possible with this kind of technology. Although the mobile application is not ideal from a
usability standpoint as mentioned before, from the standpoint of our \textit{OpenFaaS} stack however
they are also a success as they show the capability of the infrastructure of dealing with various
different devices. Our project can be therefore used as a basis of future work as there is still
room for improvement especially in the latency department. The technology of \textit{Serverless
Computing} seems to be more akin to deployment of a larger scale where many devices send data at the
same time. Optimising throughput is not the only thing, that still can be done. Security aspects
like roles and authorisation could also be imposed in order to control who and which device can call
a function.
