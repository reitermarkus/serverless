\section{Implementation}

Our task is to do \textit{IoT data analytics with serverless computing}, therefore as a first step
we started out by thinking about the underlying infrastructure. We decided on using Apache Kafka for
streaming, \textit{MongoDB} as our database and \textit{OpenFaaS} as the serverless platform.

The decision to use \textit{OpenFaaS} was made because of its ease to deploy a stack of services
using \textit{Docker Swarm}. While doing research we came across many different serverless
frameworks. \textit{OpenWhisk}, \textit{Fission} and \textit{Kubeless} just to name a few. While all
of those seem to have their benefits, none of them seemed to be as versatile as \textit{OpenFaaS}.
\textit{OpenFaaS} poses itself to have first class support for \textit{Docker Swarm} and being
\textit{Kubernetes} native. The former was particularly interesting for us as this meant that we
could test the framework without having to install any external tool except for  \textit{Docker}.
Running was as simple as cloning the \textit{OpenFaaS} repository, calling \texttt{docker swarm
init} and executing the provided initialisation script. Deploying an actual function is equally as
simple. One has the option to either deploy a function from the store through the nicely designed
\textit{OpenFaaS} gateway on port 8080 or with the preferred way, which is using the
\texttt{faas-cli}. Deploying a function from the store with it would look as follows

\begin{lstlisting}[language=bash]
$ faas store deploy figlet
\end{lstlisting}

where \texttt{figlet} is the name of the function in the store.

After gaining a grasp of how the platform works, we decided to put our own spin on it by firstly
modifying the given deploy script to our needs and porting it from \textit{Bash} to \textit{Rust}
for it to be cross platform. The next step then was to write our own configuration file for swarm
deployment, namely \texttt{deploy.yml}. This \textit{YAML} file includes the configuration for
\textit{Kafka}, \textit{Zookeeper}, \textit{MongoDB}, various services that are needed for
\textit{OpenFaaS}, a bunch of gateways for visualisation and the \textit{Kafka-Connector}. The
latter is particularly interesting, because its main purpose is to call a serverless function on a
Kafka topic change. To make a function react to a Kafka topic we can again use our example store
function \texttt{figlet}

\begin{lstlisting}[language=bash]
$ faas store deploy figlet --annotation topic="faas-request"
\end{lstlisting}

The deployment aspect is the same as before therefore the interesting part is the
\texttt{--annotation} flag, where \texttt{topic="faas-request"} is the \textit{Kafka} topic the
function is supposed to listen to and act on. We subsequently can look for the result in the logs of
the connector service.

While our technology stack for the most part was nice to work with, we still had our fair share of
problems. The most significant so far was in relation to our \textit{MongoDB} database.
\textit{MongoDB}'s API design is confusing to say the least. Questionable deprecation choices and
multiple connect interfaces are the most rampant examples. Unfortunately those were not the only
problems in that regard we ran into. Connecting through a node instance natively on the system would
work fine, however connecting through a deployed function in \textit{OpenFaaS} was not possible.
After applying various fixes the function was finally able to connect.

The next challenge will be to post to the database through the \textit{Kafka-Connector}. When that
is done we can finally focus on gathering data from IoT devices.

\newpage

\subsection{Functions}

With the OpenFaaS framework, every function consists of three parts: a function template, the
function's source code and a definition file.

Function templates are categorised by the programming language the corresponding function is written
in. At the bare minimum, a template contains a \texttt{Dockerfile} and a \texttt{template.yml}. The
\texttt{Dockerfile} has to be written in a way such that a \texttt{function} directory on the same
directory level is copied into the image. During the build step, this \texttt{function} directory
contains the source code, so depending on the programming language, it either has to be compiled or
moved to the correct location straight away. Additionally, the \texttt{Dockerfile} has to install
the \textit{OpenFaaS Watchdog}. The \textit{OpenFaaS Watchdog} is a service which is used to connect
the function to the \textit{OpenFaaS} gateway. Historically, the watchdog would pass requests to the
function via \textit{Standard Input} and the return the function's \textit{Standard Output} as the
response. With this method however, the function could not control any aspect of the \textit{HTTP}
request and response. The new version of the \textit{OpenFaaS Watchdog} offers a few more operation
modes. The first, called \texttt{http}, forwards the received request to a specified port in the
function. This means that function templates using this mode have to be written in such a way that
they include an web server. This way the function can consume the \textit{HTTP} request directly,
which also makes calling functions easier since they can use standard \textit{HTTP} methods and
status codes. Another new mode is \texttt{static}, which can be used to create very simple functions
serving static files. \cite{of-watchdog} The \texttt{template.yml} file contains the metadata for
the function. This file can be empty, i.e. all metadata is optional, but most commonly it contains
at least the \texttt{language} property used to specify the programming language the template is
meant for. \cite{openfaas-build-functions} Commonly, a \texttt{function} directory containing a
“Hello, world!” function is included in the template itself to serve as a starting point when
creating a new function from scratch.

Another part needed for building a function is a definition file containing the name of the function
and all other data needed to deploy the functions. One essential part of this definition file is the
gateway URL, which the \textit{OpenFaaS Watchdog} uses to connect to the \textit{OpenFaaS} gateway.

\begin{figure}[H]
  \begin{lstlisting}
version: 1.0
provider:
  name: openfaas
  gateway: http://127.0.0.1:8080
functions:
  filter:
    lang: rust-http
    handler: ./filter
    image: filter:latest
    environment:
      RUST_LOG: info
      write_debug: 'true'
      gateway_url: http://gateway:8080
  \end{lstlisting}
  \caption{A definition file for a Rust function called \texttt{filter}.}
  \label{fig:function-definition}
\end{figure}

In \autoref{fig:function-definition} we can see that there are two different gateway URLs. The
first one, \texttt{provider.gateway}, is used by the \texttt{faas-cli} command line tool in order to
know where to deploy the function to. The second one,
\texttt{functions.filter.environment.gateway\_url} is the URL the gateway can be reached from inside
the cluster and the one passed to the \textit{OpenFaaS Watchdog}. Furthermore, the function template
is specified using \texttt{functions.filter.lang}, the path the the function's source code is given
by \texttt{functions.filter.handler} and the name of the \textit{Docker} image is specified by
\texttt{functions.filter.image}.

Assuming the definition file shown in \autoref{fig:function-definition} is called
\texttt{filter.yml}, we can build the function using the following command:

\begin{lstlisting}[language=bash]
faas-cli build -f filter.yml
\end{lstlisting}

Once built, the function can be deployed using an equally simple command:

\begin{lstlisting}[language=bash]
faas-cli deploy -f filter.yml
\end{lstlisting}

% !TEX TS-program = xelatex
%
\documentclass{article}
\usepackage[T1]{fontenc}

\usepackage[utf8]{inputenc}
\usepackage[english]{babel}

\usepackage{xcolor}
\usepackage{listings}

\lstset{basicstyle=\ttfamily,
  breaklines = true,
  showstringspaces = false,
  commentstyle=\color{red},
  keywordstyle=\color{blue}
}

\title{IoT Data Analytics using Serverless Computing - Mobile}

\author{Markus Reiter, Michael Kaltschmid}
\date{}

\begin{document}
  \maketitle
  Another one of our tasks was to create a mobile application and fetch data from the respective device. To approach this topic we first did some research on current possibilities on creating cross platform mobile applications and the general feasibility of the task on hand. It became quite appearant early on that the desired design for the application is not possible on \textit{iOS}. \\
  The idea for the design of the app is actually rather simple. Get data from the device, no mather if the application is in foreground, background or the screen is off and the device subsequently basically being in standby.
  On \textit{Android} this can be realized easily with a so called \textit{ForegroundService}. On \textit{iOS} going for the same approach is difficult because of its restrictions on background task. While obviously applications like music players can run in the background on that platform, achieving the same for a highly battery intensive app like the one we are developing, is more or less impossible. \\
  Mainly for that reason we actually chose to go for a cross platform mobile application, so we can have a fully supported \textit{Android} version and also a \textit{iOS} version on the side. \\
  Finding the ideal tool for our job of creating a application supporting both \textit{iOS} and \textit{Android} seemed at first pretty easy. We chose to use \textit{React Native} as we both already have experience with \textit{React} itself and the approach to \textit{UI design} seemed equally elegant. \\
  Getting started with \textit{React Native} was rather fast and effortless, the same however could not be said anymore once we progressed further with the project. At first things were looking good as the UI did properly respond to data from sensors / cpu but as we added more and more data for display, the UI got more and more sluggish to the point that it could not be considered usable anymore. Unfortunately this was not the only problem. \\
  By nature the application relies heavily on \textit{native code}. The main culprit for this is the \textit{ForegroundService} on \textit{Android}. \textit{React Native} has a so called \textit{bridge} for communication between \textit{native code} and the \textit{UI}. Our application uses that bridge heavily for retrieving information from the service and also to push data to the service. \\
  Once we fully implemented the settings view and were basically finished with the application, the app started to crash constantly on \textit{Android}. Unfortunately the tooling of \textit{React Native} really does not help with debugging fatal crashes. So we had to make a conscious decision whether we want to continue with \textit{React Native} or start over with something else entirely.

\end{document}

\section{Serverless Stack}

Arguably, the main part of our thesis is the serverless stack hence it's also in the title of the
project. When doing research for our project, thinking about the serverless stack was one of the
first things we did. We came across many different serverless frameworks. \textit{OpenWhisk},
\textit{Fission} and \textit{Kubeless} just to name a few. While all of those seem to have their
benefits, none of them seemed to be as versatile as \textit{OpenFaaS}.

\textit{OpenFaaS} poses itself to have first class support for \textit{Docker Swarm} and being
\textit{Kubernetes} native. The former was particularly interesting for us as this meant that we
could test the framework without having to install any external tool except for  \textit{Docker}.

Running was as simple as cloning the \textit{OpenFaaS} repository, calling \texttt{docker swarm
init} and executing the provided initialisation script. Deploying an actual function is equally
simple. One has the option to either deploy a function from the store through the nicely designed
\textit{OpenFaaS} gateway on port~8080 or with the preferred way, which is using the
\texttt{faas-cli} command line tool.

Deploying a function from the store with it would look as follows:

\begin{lstlisting}[language=bash]
faas store deploy figlet
\end{lstlisting}

Here, \texttt{figlet} is the name of the function in the store.

After gaining a grasp of how the platform works, we decided to put our own spin on it by firstly
modifying the given deploy script to our needs and porting it from \textit{Bash} to \textit{Rust}
for it to be cross platform. The next step then was to write our own configuration file for swarm
deployment, namely \texttt{deploy.yml}. This \textit{YAML} file includes the configuration for
\textit{Kafka}, \textit{Zookeeper}, \textit{MongoDB}, various services that are needed for
\textit{OpenFaaS}, a bunch of gateways for visualisation and the \textit{Kafka-Connector}. All of
these service are \textit{Docker} images and can therefore be easily updated and extended.
\textit{Kafka-Connector} is particularly interesting, because its main purpose is to call a
serverless function on a Kafka topic change. To make a function react to a Kafka topic we can again
use our example store function \texttt{figlet}:

\begin{lstlisting}[language=bash]
faas store deploy figlet --annotation topic="faas-request"
\end{lstlisting}

The deployment aspect is the same as before, therefore the interesting part is the \\
\texttt{-{}-annotation} flag, where \texttt{topic="faas-request"} is the \textit{Kafka} topic the
function is supposed to listen to and act on. We subsequently can look for the result in the logs of
the connector service. \\
While the feature of deploying functions from the store is really nice, it is not suitable for our
use case, as we highly depend on custom functions written by ourselves.

\begin{figure}[H]
  \centering
  \adjincludegraphics[max width=\textwidth]{openfaas-dashboard}
  \caption{Picture of all functions in the \textit{OpenFaaS} dashboard}
\end{figure}

% !TEX TS-program = xelatex
%
\documentclass{article}
\usepackage[T1]{fontenc}

\usepackage[utf8]{inputenc}
\usepackage[english]{babel}

\usepackage{xcolor}
\usepackage{listings}

\lstset{basicstyle=\ttfamily,
  breaklines = true,
  showstringspaces = false,
  commentstyle=\color{red},
  keywordstyle=\color{blue}
}

\title{IoT Data Analytics using Serverless Computing - Raspberry Pi}

\author{Markus Reiter, Michael Kaltschmid}
\date{}

\begin{document}
  \maketitle

  In order for us to gather data, we decided on using multiple Raspberry Pis with different sensors attached to them. We chose the Raspberry Pi because it is backed by a huge community and the vast amount of documentation available for it. Having documentation available was essential since we wanted to write the client application for the Raspberry Pi in Rust. This way we could validate our Rust code by comparing it to example code written in other programming languages, most commonly Python or C in the case of the Raspberry Pi.

  \section{Rust on the Raspberry Pi}

  For our first “Hello, world!” programm which would run on the Raspberry Pi, we took the simplest approach at the time. We would write the code on our development machines and synchronize the code to the Raspberry Pi using \texttt{rsync} and the compile and run it via \texttt{ssh}. This worked fine at the time. Once we got to the point where we needed to add more dependencies for the various sensors and networking, compile times naturally increased to the point at which it simply wasn't feasible anymore to compile directly using the inadequate processor of the Raspberry Pi. A single build was approaching a compile time of around five minutes, so we had to start looking for alternatives.

  \section{Cross Compilation}

  Soon after we realised that compiling directly on the Raspberry Pi was not a good solution, we had to find a way to cross compile for the Raspberry Pi. This was further complicated by the fact the we were using \textit{macOS} and \textit{Windows}, so none of the pre-compiled cross compilation toolchains were an option for us.
\end{document}

\section{Serverless UI}

Another instrumental part of the project is the UI pictured in \autoref{fig:architecture-diagram}
(Web Browser \textrightarrow\ \texttt{ui}). Without any form of visualisation our whole data
gathering process would not be of much use. For this reason we decided early on what kind of front
end web development stack we would use. Since both us are familiar with \textit{React}.
\textit{React} is a \textit{JavaScript} framework famous for revolutionising the use of the
\textit{virtual DOM} to render \textit{UI} elements and to update \textit{UI} elements reactively on
UI change. However we still decided on using \textit{MarkoJS} \cite{marko}.

\subsection{MarkoJS}

\textit{MarkoJS} shares many of the same benefits as \textit{React} with some added flexibility like
\textit{concise HTML syntax} which makes the whole \textit{markup} more readable and easier to
write. It also has the ability to use conventional control flow structures like \textit{if} or
\textit{for} directly in the \textit{markup}.

\subsection{Babel}

\textit{MarkoJS} is most and foremost a \textit{JavaScript} framework and due to the nature of the
before mentioned features, \textit{transpiling} is inevitable. \textit{Transpiling} means to
transform modern possibly unsupported \textit{JavaScript} into valid \textit{ECMAScript 5} compliant
code that any browser can understand. For this process we currently use the industry standard
technology \textit{Babel} \cite{babel}. With \textit{Babel} \textit{transpiling} is rather easy. All
necessary definitions are in a \textit{webpack.config.babel.js} file which brings us to the next
essential tool \textit{Webpack}.

\subsection{Webpack}
\label{sec:webpack}

“Webpack is a module bundler. Its main purpose is to bundle JavaScript files for usage in a browser,
yet it is also capable of transforming, bundling, or packaging just about any resource or asset.”
\cite{webpack}

With that being said \textit{Webpack} can to some extend be considered as the main part of the whole
\whitelist{frontend stack} as it is responsible for orchestrating \textit{transpiling} of
\textit{MarkoJS} files with the help of \textit{Babel}. It is also responsible for providing
\textit{Webpack Dev Server}, a server that reloads on file change. Furthermore \textit{Webpack} runs
all files through certain optimisation plugins on release build, which can bring down the size of
the \textit{code bundle} quite considerably. \textit{Webpack} is also able to transform
\textit{SASS} into \textit{CSS}.

\subsection{SASS}

\textit{SASS} \cite{sass} is a superset of \textit{CSS} with many additional features like
variables, functions, nesting and exporting/importing files. We make heavy use of those features to
structure and modularise our \textit{CSS} code.

\subsection{UI}
\label{sec:ui}

The \textit{UI} itself uses a basic layout where all sensor devices registered in the database are
listed on the side in a so called \textit{sidebar}. The user then can click on each individual
device to open a view with detailed graphs of sensor data of that device. The categories of graphs
are different according to device group. For example \textit{IoT devices} will have different
sensors compared to phones. After a device is selected, the user can then specify the begin, end
time interval and the amount of time slices according to which sensor data of that device will be
filtered. By default the time span is limited to 24~hours and to 24~time slices. A time slice in our
case is a division of a time interval into equally long time units. To get a single value per time
slice we average all sensor values of a specific sensor between two time slices. Those single values
will then be displayed in the graphs. In \ref{fig:ui} the whole UI can be seen. The sidebar is on
the left, where all registered devices are listed. In this case the device “Samsung Galaxy S7” is
selected and therefore all graphs for this specific device will be shown.

\begin{figure}[H]
  \centering
  \adjincludegraphics[max width=\textwidth]{ui-2}
  \caption{Serverless UI - The whole UI}
  \label{fig:ui}
\end{figure}

\begin{figure}[H]
  \centering
  \adjincludegraphics[max width=\textwidth]{ui-graphs}
  \caption{Serverless UI - Graphs: More graphs for the selected device “Samsung Galaxy S7”.}
\end{figure}

\begin{figure}[H]
  \centering
  \adjincludegraphics[max width=\textwidth]{ui-slider}
  \caption{Serverless UI - Slider}
\end{figure}

\subsection{Azure Pipelines}

In order to properly verify our results we used \textit{Azure Pipelines} as our continues integration service
of choice. \textit{Azure Pipelines} offers 10 parallel jobs with unlimited time per job for
open source projects. \cite{azure-pipelines-devop}

In our case we used the \textit{CI} platform for 6 jobs, that are \textit{app}, \textit{rpi}
\textit{functions}, \textit{ui} and \textit{tex}.

In \textit{app} we build and test our \textit{Flutter mobile application}. For the \textit{vmImage}
the job will run on we have to choose \textit{macOS} in order to build the application for
\textit{Android} and \textit{iOS} since \textit{Apple} restricts building of \textit{iOS apps} to
\textit{macOS}.

The \textit{rpi} job is responsible for building and testing the application, that will be deployed
on the \textit{Rasperry Pi} in order to measure sensor data. The program is written in
\textit{Rust}. Because \textit{Rust} is cross platform, a conventional \textit{Ubuntu} image is
sufficient. Unfortunately \textit{Azure Pipelines} virtual machines do not have \textit{Rust}
installed by default, therefore we have to rely on a template provided by the \textit{Rust Cargo
team} \cite{rust-cargo}. This \textit{Azure Pipelines} job however is not trivial and requires
multiple steps. \textit{Rasperry Pis} use \textit{ARM} as there instruction set and therefore differ
from the instruction set used in the virtual machines of \textit{Azure}. A simple compile of the
application is for this reason not possible. In order to run the program on the \textit{Rasperry
PI}, the application actually has to be cross compiled with the correct toolchain
\textit{armv7-unknown-linux-gnueabi}.

With the \textit{functions} job we build all the \textit{OpenFaaS} functions in the pipeline. It is
also arguably the most complex job since it requires multiple steps. First off in order to deploy
functions in \textit{OpenFaaS} the \textit{OpenFaaS CLI} is required and for that reason it has to
be installed as well. Since our whole deploy script is written \textit{Ruby} or more specifically in
\textit{Rake} we also need to install that. The next step is the most important one of this job, the
actual build of the functions. Compiling \textit{Rust} is rather slow and therefore occupies about
20 minutes on \textit{Azure} for building all functions. Build those functions, however is not only
time consuming on the \textit{CI} platform. Deploying the whole stack on our test device, the
\textit{Intel NUC} also takes some time, mainly because of function compiling. For this reason we
have a further step in the \textit{functions} job, which pushes all images to the \textit{GitHub
Package registry}. Every subsequent fresh deploy then only needs to pull the images from the
registry instead of building them each time.

\begin{figure}[H]
  \centering
  \adjincludegraphics[max width=\textwidth]{github-registry}
  \caption{Images of all functions in \textit{GitHub Package registry} on \url{https://github.com/reitermarkus/serverless/packages}}
\end{figure}

