\chapter{Introduction}
\label{sec:introduction}

In recent years, the number of IoT (Internet of Things) devices has been growing continuously.
Naturally, all of these devices generate a lot of data. This usually happens asynchronously, which
means that a normal server processing this data would either be under or overutilised. A well known
method to deal with problem is to use the edge-fog-cloud approach to process data earlier, either on
an edge device itself or on a fog device instead of doing all processing in the cloud. This can have
noticable performance benefits. This is where serverless computing comes into play. In contrast to
traditional software development where the software is centred around a single server delivering all
of the functionality, serverless computing is based around the idea of splitting this functionality
into separate functions. This is especially beneficial when dealing with many IoT devices, since
functions can be scaled individually according to the current utilisation and can be started on
demand. This results in more efficient usage of resources and improved modularity.

Today IoT devices can be found in household appliances like washing machines, fridges, televisions,
garage doors and many more. For this thesis, their application for the purpose of data collection
was our primary focus. A big advantage to using IoT devices is the low cost of some of the
do-it-yourself and open-source solutions, which can be achieved with a micro-controller such as the
Raspberry Pi and readily available off-the-shelf sensors.

In this thesis, we explore the possibility of combining both serverless computing and IoT for the
purpose of analysing sensor data. More specifically, we want to test the limits of cheap IoT
hardware in a serverless context. Furthermore, we want to subsequently visualise the collected data
in an intuitive user interface that can be displayed on any device.

The end result of this thesis is a system which can be used to collect sensor data from IoT devices
using a serverless architecture and an accompanying web interface to visualise the collected data.
In the following sections we explain the technologies used, a more detailed explanation about the
motivation for using serverless computing as well as how the different parts of our project were
implemented. Finally, we provide some results on how our system performs when running on affordable
hardware.
