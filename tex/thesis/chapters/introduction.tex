\chapter{Introduction}
\label{sec:introduction}

In recent years, the number of IoT (Internet of Things) devices has been growing continuously.
Naturally, all of these devices generate a lot of data. This usually happens asynchronously, which
means that a normal server processing this data would either be under or overutilised most of the time. A well known
method to deal with this kind of problem is to use the \textit{edge-fog-cloud} approach to process
data earlier, either on an edge device or on a fog device instead of doing all processing in
the cloud, which can have noticeable performance benefits. This is where \textit{serverless
computing} comes into play. In contrast to traditional software development where the software is
centred around a single server delivering all of the functionality, \textit{serverless computing} is
based around the idea of splitting this functionality into separate functions, which is especially
beneficial when dealing with many IoT devices. Functions can be scaled individually according
to the current utilisation and can be started on demand, resulting in more efficient usage of
resources and improved modularity.

Today IoT devices can be found in household appliances like washing machines, fridges, televisions,
garage doors and many more. For this thesis, their application for the purpose of data collection
was our primary focus. A big advantage to using IoT devices is the low cost of some of the
do-it-yourself and open source solutions, which can be achieved with a
micro-controller such as the \textit{Raspberry Pi} and readily available
\whitelist{off-the-shelf} sensors.

In this thesis, we explore the possibility of combining both \textit{serverless computing} and IoT
for the purpose of analysing sensor data. More specifically, we want to test the limits of cheap IoT
hardware in a \textit{serverless} context. In order to achieve this, we decided to use the
\textit{OpenFaaS} framework to realise the platform for hosting functions. Furthermore,
\textit{Apache Kafka} is used to manage all incoming requests from devices. \textit{Kafka} also
handles the forwarding of requests to functions. For persistence of sensor data, we are using the
\textit{NoSQL} database \textit{MongoDB}, which is of great help when confronted with data from
IoT devices. Having a stack like this also means dealing with a lot of configuration. To
aid this process, we also developed a \textit{Rake} script which makes everything runnable with only
one command. And finally, we are using \textit{Azure Pipelines} as our verification tool to test all
parts of our stack and push our \textit{serverless functions} to an online registry, which makes
deployment even faster. Furthermore, we want to subsequently visualise the collected data in an
intuitive user interface that can be displayed on any device.

The end result of this thesis is a system which can be used to collect sensor data from IoT devices
using a \textit{serverless} architecture and an accompanying web interface to visualise the
collected data. In the following sections we explain the technologies we used and go into more detail about the motivation for using \textit{serverless computing} as well as how the
different parts of our project were implemented.

Ultimately, we present results including benchmarks for running our stack on affordable hardware,
showing just how cost-effective and yet powerful a serverless stack can be.
