\section{Introduction}

\subsection{Serverless Computing}

In recent years, serverless computing has been gaining traction as an alternative to the classic
monolithic approach to developing software. In contrast to traditional software development where
the software is centred around a single server delivering all of the functionality, serverless
computing is based around the idea of splitting this functionality into separate functions. Each of
these functions can then be deployed on a serverless stack. Depending on the stack, the functions
will only be spun up on demand when they are called and killed when they are not needed anymore.
This results in more efficient usage of resources and improved modularity.

The name “serverless computing” is not quite fitting as there is still a server that is hosting the
serverless stack, however, unlike in the traditional sense, the user does not have to set up all
the services themselves. For this reason most big cloud providers (AWS, Azure, Google, etc.) offer
services that mimic serverless computing. While those services might be easier to use and less work
to set up, a lot of flexibility is lost in availability of programming languages and in our case
more importantly in the flexibility of hardware resource distribution.

\subsection{IoT (Internet of Things)}

Similar to serverless computing, “Internet of Things” devices increased in popularity in the last
couple of years. IoT devices are devices which are connected via the Internet either for the
purpose of collecting data from sensors or in order for them to control or be controlled by other
IoT devices. Today IoT devices can be found in household appliances like washing machines, fridges,
televisions, garage doors and many more.

For this thesis, their application for the purpose of data collection was our primary
focus. Another big advantage to using IoT devices is the low cost of some of the do-it-yourself and
open-source solutions, which can be achieved with a micro-controller such as the Raspberry Pi and
readily available off-the-shelf sensors.

\subsection{Data Analytics}

In this thesis, we explore the possibility of combining both serverless computing and IoT for the
purpose of analysing sensor data. More specifically, we want to test the limits of cheap IoT
hardware in a serverless context. Furthermore, we want to subsequently visualise the collected
data in an intuitive user interface that can be displayed on any device.
