\subsection{Azure Pipelines}

In order to properly verify our results we used \textit{Azure Pipelines} as our continues
integration service of choice. \textit{Azure Pipelines} offers 10 parallel jobs with unlimited time
per job for open source projects. \cite{azure-pipelines-devop}

In our case we used the \textit{CI} platform for 6 jobs, that are \textit{app}, \textit{rpi}
\textit{functions}, \textit{ui}, \textit{spellcheck} and \textit{tex}.

\subsubsection{The \texttt{app} job}

In \textit{app} we build and test our \textit{Flutter mobile application}. For the \textit{vmImage}
the job will run on we have to choose \textit{macOS} in order to build the application for
\textit{Android} and \textit{iOS} since \textit{Apple} restricts building of \textit{iOS apps} to
\textit{macOS}.

\subsubsection{The \texttt{rpi} job}

The \textit{rpi} job is responsible for building and testing the application, that will be deployed
on the \textit{Rasperry Pi} in order to measure sensor data. The program is written in
\textit{Rust}. Because \textit{Rust} is cross platform, a conventional \textit{Ubuntu} image is
sufficient. Unfortunately \textit{Azure Pipelines} virtual machines do not have \textit{Rust}
installed by default, therefore we have to rely on a template provided by the \textit{Rust Cargo
team} \cite{rust-cargo}. This \textit{Azure Pipelines} job however is not trivial and requires
multiple steps. \textit{Rasperry Pis} use \textit{ARM} as there instruction set and therefore differ
from the instruction set used in the virtual machines of \textit{Azure}. A simple compile of the
application is for this reason not possible. In order to run the program on the \textit{Rasperry
PI}, the application actually has to be cross compiled with the correct toolchain
\textit{armv7-unknown-linux-gnueabi}.

\subsubsection{The \texttt{function} job}
\label{sec:azure-function}

With the \textit{functions} job we build all the \textit{OpenFaaS} functions in the pipeline. It is
also arguably the most complex job since it requires multiple steps. First off in order to deploy
functions in \textit{OpenFaaS} the \textit{OpenFaaS CLI} is required and for that reason it has to
be installed as well. Since our whole deploy script is written \textit{Ruby} or more specifically in
\textit{Rake} we also need to install that. The next step is the most important one of this job, the
actual build of the functions. Compiling \textit{Rust} is rather slow and therefore occupies about
20 minutes on \textit{Azure} for building all functions. Build those functions, however is not only
time consuming on the \textit{CI} platform. Deploying the whole stack on our test device, the
\textit{Intel NUC} also takes some time, mainly because of function compiling. For this reason we
have a further step in the \textit{functions} job, which pushes all images to the \textit{GitHub
Package registry}. Every subsequent fresh deploy then only needs to pull the images from the
registry instead of building them each time.

\begin{figure}[H]
  \centering
  \adjincludegraphics[max width=\textwidth]{github-registry}
  \caption{Images of all functions in \textit{GitHub Package registry} on \url{https://github.com/reitermarkus/serverless/packages}}
\end{figure}

\subsubsection{The \texttt{UI} job}

The \textit{UI} job is among the fastest running jobs as it only has to build the website with
\textit{WebPack}.

\subsubsection{The \texttt{spellcheck} job}

On the other hand the \textit{spellcheck} job is much more interesting and also is generally way
more handy. The main purpose of it is, to check if the spelling conforms with \textit{aspell's}
British English dictionary. Unfortunately in our case the default output of \textit{aspell} is too
basic and for this reason we need a script (modified version of \cite{aspell-awk}) to make it more
useful. By default \textit{aspell} only outputs errors without any more context. However we would
like to have it a bit more verbose. Because \textit{aspell's} support for \textit{LaTeX} is not
quite as good as we would like it to be, we first have to remove all content inside of
\textit{lstlisting} commands. The next step is to then check all \textit{LaTeX} documents with
\textit{aspell} and finally we pipe that output into \textit{Awk} to get line number information.
The script then essentially fails if there is an output and passes if there is none (no spelling
mistakes).

\begin{figure}[H]
  \begin{lstlisting}[language=bash, basicstyle=\small\ttfamily]
    thesis/chapters/background.tex
    ----
    87:32 ce
    90:19 Cricle
    91:19 Anageles

    thesis/chapters/implementation.tex
    ----
    78:8 openfaas
    82:4 lang
    82:15 http
    88:14 url
    104:25 yml
    108:26 yml

    thesis/chapters/implementation/ui.tex
    ----
    42:23 timespan

    ##[error]Bash exited with code '1'.
  \end{lstlisting}
  \caption{Modified output of \textit{aspell}. The script fails as there are spelling mistakes.}
\end{figure}

\subsubsection{The \texttt{tex} job}

The last function is \textit{tex}, which builds the initial presentation and the thesis on each
commit. To accomplish this we use the \textit{mirisbowring/texlive\_ctan\_full Docker image}. This
image provides as with a complete \textit{LaTeX} environment without much configuration hassle.
After a successful build we make use of the \textit{continues delivery} aspect of \textit{Azure
Pipelines} by hosting the compiled \textit{LaTeX} document as an artefact on \textit{Azure}.

\begin{figure}[H]
  \centering
  \adjincludegraphics[max width=\textwidth]{artifact}
  \caption{\textit{Azure Pipeline} artefact on \url{https://dev.azure.com/reitermarkus/serverless/}}
\end{figure}
