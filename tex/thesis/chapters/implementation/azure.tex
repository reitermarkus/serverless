\section{Azure Pipelines}

In order to properly verify our results, we used \textit{Azure Pipelines} as our continuous
integration service of choice. \textit{Azure Pipelines} offers 10~parallel jobs with unlimited time
per job for open source projects. \cite{azure-pipelines-devop}

In our case we used the \textit{CI} platform for 6~jobs, that are \texttt{app}, \texttt{rpi}
\texttt{functions}, \texttt{ui}, \texttt{spellcheck} and \texttt{tex}.

\subsection{The \texttt{app} job}

In \texttt{app}, we build and test our \textit{Flutter} mobile application. For the \texttt{vmImage}
the job will run on, we chose \textit{macOS} in order to build the application for \textit{Android}
as well as \textit{iOS} since \textit{Apple} restricts building of \textit{iOS} apps to
\textit{macOS}.

\subsection{The \texttt{rpi} job}

The \texttt{rpi} job is responsible for building and testing the application that will be deployed
on the \textit{Rasperry Pi} in order to measure sensor data. The program is written in
\textit{Rust}. Because \textit{Rust} is cross platform, a conventional \textit{Ubuntu} image is
sufficient. Unfortunately \textit{Azure Pipelines} virtual machines do not have \textit{Rust}
installed by default, therefore we have to rely on a template provided by the \textit{Rust Cargo
team} \cite{rust-cargo}. This \textit{Azure Pipelines} job however is not trivial and requires
multiple steps. \textit{Rasperry Pis} use \textit{ARM} as their instruction set and therefore differ
from the instruction set used in the virtual machines of \textit{Azure}. A simple compilation of the
application is not possible for this reason. In order to run the program on the \textit{Rasperry
PI}, the application actually has to be cross compiled with the correct toolchain,
\texttt{armv7-unknown-linux-gnueabi}.

\subsection{The \texttt{functions} job}
\label{sec:azure-function}

With the \texttt{functions} job, we build all the \textit{OpenFaaS} functions in the pipeline. It is
also arguably the most complex job since it requires multiple steps. First off in order to deploy
functions in \textit{OpenFaaS}, the \textit{OpenFaaS CLI} \cite{faas-cli} is required and for that
reason it has to be installed as well. Since our whole deploy script is written in \textit{Ruby} or
more specifically in \textit{Rake}, we also need to install that. The next step is the most important
one of this job, the actual building of the functions. Compiling \textit{Rust} is rather slow and
therefore occupies about 20~minutes on \textit{Azure} for building all functions. Building those
functions, however is not only time consuming on the \textit{CI} platform. Deploying the whole stack
on our test device, the \textit{Intel NUC} also takes some time, mainly due to compiling functions.
For this reason we have a further step in the \texttt{functions} job, which pushes all images to the
\textit{GitHub Package Registry} \cite{github-registry}. Every subsequent fresh deployment on the
\textit{NUC} then only needs to pull the images from the registry instead of building them each time.

\begin{figure}[H]
  \centering
  \adjincludegraphics[max width=\textwidth]{github-registry}
  \caption{Images of all functions in \textit{GitHub Package Registry} on \\
    \url{https://github.com/reitermarkus/serverless/packages}}
\end{figure}

\subsection{The \texttt{ui} job}

The \texttt{ui} job is among the shortest running jobs as it only has to build the website with
\textit{Webpack}.

\subsection{The \texttt{spellcheck} job}

On the other hand the \texttt{spellcheck} job is much more interesting. The main purpose of it is
to check if the spelling conforms with \texttt{aspell}'s British English dictionary. Unfortunately
in our case the default output of \textit{aspell} is too basic and for this reason we need a script
(modified version of \cite{aspell-awk}) to make it more useful. By default \texttt{aspell} only
outputs errors without any more context. However we would like to have it a bit more verbose.
Because \textit{aspell}'s support for \textit{LaTeX} is not quite as good as we would like it to be,
we first have to remove all content inside of \texttt{lstlisting} commands. The next step is to
check all \textit{LaTeX} documents with \texttt{aspell} and finally pipe that output into
\texttt{awk} to get line number information. The script then fails if there is any output
and passes if there is none (no spelling mistakes). In \autoref{fig:spellcheck} you can see an
example of a failed check.

\begin{figure}[H]
  \centering
  \begin{lstlisting}[language=bash, basicstyle=\footnotesize\ttfamily]
thesis/chapters/background.tex
----
90:19 Cricle
91:19 Anageles

thesis/chapters/implementation.tex
----
78:8 openfaas
82:4 lang
88:14 url

thesis/chapters/implementation/ui.tex
----
42:23 timespan

##[error]Bash exited with code '1'.
  \end{lstlisting}
  \caption{Modified output of \texttt{aspell}. The script fails, as there are spelling mistakes.}
  \label{fig:spellcheck}
\end{figure}

\subsection{The \texttt{tex} job}

The last job is \texttt{tex}, which builds the presentations and the thesis on each
commit. To accomplish this we use the \texttt{mirisbowring/texlive\_ctan\_full} \textit{Docker} image
\cite{tex-hub}. This image provides us with a complete \textit{LaTeX} environment without much
configuration hassle. After a successful build we make use of the \textit{continuous delivery} aspect
of \textit{Azure Pipelines} by hosting the compiled \textit{LaTeX} document as an artefact on
\textit{Azure}.

\begin{figure}[H]
  \centering
  \adjincludegraphics[max width=\textwidth]{artifact}
  \caption{\textit{Azure Pipeline} artefact on \\
    \url{https://dev.azure.com/reitermarkus/serverless/_build?definitionId=2&_a=summary}}
\end{figure}
