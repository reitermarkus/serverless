\section{Kafka}

In our serverless stack, \textit{Apache Kafka} is the first point of contact for our edge devices
(Raspberry Pis and Smartphones). Additionally, the stack includes \textit{Apache Zookeeper}
(explained in \autoref{sec:background-zookeeper}), which \textit{Kafka} uses to register itself.
This way, other services can connect to Zookeeper instead of connecting to Kafka directly, which is
needed when running multiple \textit{Kafka} instances in a cluster.

Since we wanted to use a unified API to post data to Kafka, our stack also includes \textit{Kafka
Rest}, which provides a \textit{ReST} (\textit{Representational State Transfer}) API endpoint for
the \textit{Kafka} core APIs explained in \autoref{sec:background-kafka}. The \textit{ReST} API is
subsequently used by the mobile application as well as the Raspberry Pi application to post sensor
data.

Another building block we need in our stack is the \textit{OpenFaaS Kafka Connector}, which is used
to forward messages from one or multiple topics to the \textit{OpenFaaS Gateway}. In our specific
case, we have two topics that are forwarded: \texttt{sensor} and \texttt{register-device}.

One more useful thing we have included in our stack is the \textit{Kafka Topics UI}. The
\textit{Topics UI} provides an overview of all topics and a table of all messages in a given topic.
Having this information is very helpful when debugging in order to know whether a device is posting
data at all or to the correct topic.
