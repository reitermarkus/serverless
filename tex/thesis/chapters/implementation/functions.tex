\subsection{Functions}

With the OpenFaaS framework, every function consists of three parts: a function template, the
function's source code and a definition file.

Function templates are categorised by the programming language the corresponding function is written
in. At the bare minimum, a template contains a \texttt{Dockerfile} and a \texttt{template.yml}. The
\texttt{Dockerfile} has to be written in a way such that a \texttt{function} directory on the same
directory level is copied into the image. During the build step, this \texttt{function} directory
contains the source code, so depending on the programming language, it either has to be compiled or
moved to the correct location straight away. Additionally, the \texttt{Dockerfile} has to install
the \textit{OpenFaaS Watchdog}. The \textit{OpenFaaS Watchdog} is a service which is used to connect
the function to the \textit{OpenFaaS} gateway. Historically, the watchdog would pass requests to the
function via \textit{Standard Input} and the return the function's \textit{Standard Output} as the
response. With this method however, the function could not control any aspect of the \textit{HTTP}
request and response. The new version of the \textit{OpenFaaS Watchdog} offers a few more operation
modes. The first, called \texttt{http}, forwards the received request to a specified port in the
function. This means that function templates using this mode have to be written in such a way that
they include an web server. This way the function can consume the \textit{HTTP} request directly,
which also makes calling functions easier since they can use standard \textit{HTTP} methods and
status codes. Another new mode is \texttt{static}, which can be used to create very simple functions
serving static files. \cite{of-watchdog} The \texttt{template.yml} file contains the metadata for
the function. This file can be empty, i.e. all metadata is optional, but most commonly it contains
at least the \texttt{language} property used to specify the programming language the template is
meant for. \cite{openfaas-build-functions} Commonly, a \texttt{function} directory containing a
“Hello, world!” function is included in the template itself to serve as a starting point when
creating a new function from scratch.

Another part needed for building a function is a definition file containing the name of the function
and all other data needed to deploy the functions. One essential part of this definition file is the
gateway URL, which the \textit{OpenFaaS Watchdog} uses to connect to the \textit{OpenFaaS} gateway.

\begin{figure}[H]
  \centering
  \begin{lstlisting}
version: 1.0
provider:
  name: openfaas
  gateway: http://127.0.0.1:8080
functions:
  filter:
    lang: rust-http
    handler: ./filter
    image: filter:latest
    environment:
      RUST_LOG: info
      write_debug: 'true'
      gateway_url: http://gateway:8080
  \end{lstlisting}
  \caption{A definition file for a Rust function called \texttt{filter}.}
  \label{fig:function-definition}
\end{figure}

In \autoref{fig:function-definition} we can see that there are two different gateway URLs. The
first one, \texttt{provider.gateway}, is used by the \texttt{faas-cli} command line tool in order to
know where to deploy the function to. The second one,
\texttt{functions.filter.environment.gateway\_url} is the URL the gateway can be reached from inside
the cluster and the one passed to the \textit{OpenFaaS Watchdog}. Furthermore, the function template
is specified using \texttt{functions.filter.lang}, the path the the function's source code is given
by \\ \texttt{functions.filter.handler} and the name of the \textit{Docker} image is specified by \\
\texttt{functions.filter.image}.

Assuming the definition file shown in \autoref{fig:function-definition} is called
\texttt{filter.yml}, we can build the function using the following command:

\begin{lstlisting}[language=bash]
faas-cli build -f filter.yml
\end{lstlisting}

Once built, the function can be deployed using an equally simple command:

\begin{lstlisting}[language=bash]
faas-cli deploy -f filter.yml
\end{lstlisting}

We chose to write all of our functions in Rust. When we first started, there was not yet an official
\textit{OpenFaaS} function template for Rust, so we had to create our own. For our custom template,
we used the builder pattern our \texttt{Dockerfile}. In our case this meant we would first compile
the Rust function using the official \textit{Docker} image for \textit{Rust}. This image is based on
\textit{Debian}, which means that we could then copy the compiled function into a bare
\textit{Debian} image which doesn't include the \textit{Rust} compiler, therefore saving space. This
resulted in each function image being around \SI{90}{\mega\byte} in size. Still, this was too much
for a simple function in our opinion.

It is possible to write a \texttt{Dockerfile} without using any base image, using the \texttt{FROM
scratch} directive. For this to work, however, we need to generate a statically compiled version of
our functions since there is no operating system providing libraries. By extension, this meant we
had to replace our compile step, since the official \texttt{Rust} docker image does not provide
support for creating statically linked binaries. Thankfully, there exists a \textit{Docker} image
called \texttt{rust-musl-builder} \cite{rust-musl-builder} for this exact reason. Using the
\texttt{rust-musl-builder} image, we could reduce the average image size for our functions to around
\SI{15}{\mega\byte}, a sixth of their original size.