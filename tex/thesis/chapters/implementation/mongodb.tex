\section{MongoDB}

While our technology stack for the most part was nice to work with, we still had our fair share of
problems. One of the most significant one was in relation to our \textit{MongoDB} database.
\textit{MongoDB’s API design} is confusing to say the least. Questionable deprecation choices and
multiple connect interfaces are the most rampant examples. Once we got past those issues and got a
grasp of how it all works, \textit{MongoDB} does seem to be a decent choice for our task at hand,
mainly because of its JSON-like document architecture. Posting the same \textit{JSON} structure
uniformly from all devices to the \textit{Kafka RESt endpoint} is a fairly trivial task. The
architecture of our \texttt{sensordata} database is as follows: Every sensor type has its own
collection with documents containing the device ID of the sender, the time stamp of when the sensor
data was collected and a unique ID per entry. Those documents differ depending on whether the sensor
only emits one value or multiple values. An example for a single value sensor document would be
\textit{pressure} and an example with multiple values would be \textit{orientation}, where the
sensor collects the value of the x, y and z axis. We also need one collection for all the devices
that are registered, called \texttt{devices}, which contains documents with a unique ID per entry,
the name of the device and the supported sensor types. Devices have to be registered in order for
the \textit{UI} to know which graphs can be displayed for each device.

\begin{code}[H]
  \centering
  \begin{lstlisting}[language=mongo]
{
  _id: 'c7704c421d7491ec',
  name: 'Samsung Galaxy S7',
  data_types: [
      'cpu_frequency',
      'proximity',
      'acceleration',
      'rotation',
      'pressure',
      'orientation',
      'gravity',
      'illuminance',
      'rotation_rate',
      'rotation_rate_uncalibrated',
      'magnetic_field',
      'magnetic_field_uncalibrated'
  ]
}
  \end{lstlisting}
  \caption{Registered device “Samsung Galaxy S7” with all its supported data and sensor types.}
\end{code}
