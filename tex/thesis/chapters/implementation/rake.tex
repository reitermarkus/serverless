\subsection{Rakefile}

“Ruby is a dynamic, open source programming language with a focus on simplicity and productivity. It
has an elegant syntax that is natural to read and easy to write.” \cite{ruby} Ruby is also
\textit{cross-platform} and has an optional build system with the name \textit{Rake} which is
similar in terms of functionality to \textit{Make}, with the difference that the code for its tasks
is written in \textit{Ruby}. And with that being said, \textit{Ruby} and \textit{Rake} are the ideal
candidates for our deployment script. The aim for our script was that with one command the whole
stack would be correctly configured and deployed. This can be done with a simple
\lstinline{rake deploy}.

The command first checks if the \textit{Docker Swarm} is active. The swarm has to be active, in
order for functions to be able to be deployed. In the next step a loop goes over all functions and
invokes the \texttt{faas-cli} for the actual deployment of the function. If the swarm however is not
yet ready, \lstinline{rake deploy} invokes a subtask that orchestrates \textit{Docker} to set up a
swarm with the necessary credentials and secrets. In the same step, all needed environment variables
for \textit{MongoDB} and \textit{Kafka} are set and finally the actual deployment of the stack
commences and all services are initialised. The final step of the \lstinline{rake deploy} task
depends on the platform the script is executed. On \textit{Windows} \lstinline{rake db:restore} will
be invoked, which tries to restore the \textit{MongoDB} database from a backup, because
\textit{Docker} volumes do not behave the way they should in combination with \textit{MongoDB} on
that platform.

During the development of our functions, we had to rebuild functions frequently in order to test
their correctness. To aid this process we have the \lstinline{rake build:functions} \textit{Rake}
task. When simply called without parameters, it builds all functions with the \texttt{faas-cli
build} command. Internally, \texttt{faas-cli} then builds a \textit{Docker} image among other
things. \lstinline{rake build:functions} with parameters on the other hand can be used to build only
a single function. As an example, we can call \lstinline{rake build:functions[ui]} to only build the
\textit{UI} function.

Because we like to upload our updated \textit{Docker} images on \textit{GitHub Container Registry}
for more convenience when deploying the stack, we decided to automate this process with a
\lstinline{rake build:push} \textit{Rake} task. It iterates over all functions in a very similar way
to how the \lstinline{rake build:functions} task does, with the exception of invoking
\texttt{faas-cli} with the \textit{push} instead of the \textit{build} subcommand. The
\lstinline{rake build:push} task is also used in the \textit{Azure Pipelines} \textit{functions} job
explained in \autoref{sec:azure-function}.

We have also some \textit{Rake} tasks that are not directly related to the \textit{OpenFaaS} stack.
Among those is the \lstinline{rake tex} task, which invokes \texttt{latexmk} for both the
presentation and thesis. Finally we have the \textit{Rake} namespace \lstinline{db}, in which we
have tasks to help us restore and dump our \textit{MongoDB} database.
