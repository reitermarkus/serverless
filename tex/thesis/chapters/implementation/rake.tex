\section{\whitelist{Rakefile}}

The aim for our deployment script was that with one command, the whole stack would be correctly
configured and deployed. Using a \textit{Rakefile}, this can be done with a simple
\lstinline{rake deploy}.

The command first checks if the \textit{Docker Swarm} is active since the swarm has to be active in
order for functions to be able to be deployed. In the next step, a loop goes over all functions and
invokes the \texttt{faas-cli} for the actual deployment of the function. If the swarm however is not
yet ready, \lstinline{rake deploy} invokes a subtask that orchestrates \textit{Docker} to set up a
new swarm with the necessary credentials and secrets. In the same step, all needed environment variables
for \textit{MongoDB} and \textit{Kafka} are set and finally the actual deployment of the stack
commences and all services are initialised. The final step of the \lstinline{rake deploy} task
depends on the platform the script is executed. On \textit{Windows} \lstinline{rake db:restore} will
be invoked, which restores the \textit{MongoDB} database from a backup if one exists, because
\textit{Docker} volumes do not behave the way they should in combination with \textit{MongoDB} on
that platform.

During the development of our functions, we had to rebuild functions frequently in order to test
their correctness. To aid this process we have the \texttt{build:functions} \textit{Rake}
task. When simply called without parameters, it builds all functions with the \texttt{faas-cli
build} command. Internally, \texttt{faas-cli} then builds a \textit{Docker} image among other
things. Calling \lstinline{rake build:functions} with parameters on the other hand can be used to
build only a single function. As an example, we can call \lstinline{rake build:functions[ui]} to only
build the \textit{UI} function.

Because we like to upload our updated \textit{Docker} images on \textit{GitHub Container Registry}
for more convenience when deploying the stack, we decided to automate this process with a
\texttt{build:push} \textit{Rake} task. It iterates over all functions in a very similar way
to how the \texttt{build:functions} task does, with the exception of invoking
\texttt{faas-cli} with the \texttt{push} instead of the \texttt{build} subcommand. The
\texttt{build:push} task is also used in the \textit{Azure Pipelines} \texttt{functions} job
explained in \autoref{sec:azure-function}.

We have also some \textit{Rake} tasks that are not directly related to the \textit{OpenFaaS} stack.
Among those is \lstinline{rake tex}, which invokes \texttt{latexmk} for both the
presentation and thesis. Finally we have the \textit{Rake} namespace \texttt{db}, in which we
have tasks to help us restore and dump our \textit{MongoDB} database.
