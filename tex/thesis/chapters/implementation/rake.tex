\subsection{Rakefile}

“Ruby is a dynamic, open source programming language with a focus on simplicity and productivity. It
has an elegant syntax that is natural to read and easy to write.” \cite{ruby} Ruby is also
\textit{cross-platform} and has an optional build system with the name \textit{Rake}, that is
similar in usability to \textit{Make} with the difference that the code for its tasks is still
\textit{Ruby}. And with that being said \textit{Ruby} and \textit{Rake} are the ideal candidates for
our deployment script. The aim for our script was that with one command the whole stack would be
correctly configured and deployed. This can be done with a simple \lstinline{rake deploy}.

The command first checks if the \textit{docker swarm} is active. The \textit{swarm} has to be
active, in order that functions even can be deployed. In the next step a loop goes over all
functions and invokes the \textit{faas-cli} for the actual deployment of the function. If the swarm
however is indeed not ready \lstinline{rake deploy} invokes a \textit{sub-task} that orchestrates
\textit{docker} to setup a \textit{swarm} with necessary credentials and secrets. In the same step
all kind of environment variables have to be set for \textit{MongoDB}, \textit{Kafka} and finally
the actual deployment of the stack commences and all services will be initialised. The final step of
the \lstinline{rake deploy} task depends on the platform the script is executed. On \textit{Windows}
\lstinline{rake db:restore} will be invoked, which tries to restore the \textit{MongoDB} database,
because \textit{docker volumes} do not behave the way they should in correlation to\textit{MongoDB}
on that platform.