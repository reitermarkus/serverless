\begin{frame}{Functions -- Overview}
  \begin{itemize}
    \item can be written in any language
    \item built and deployed using \texttt{faas-cli}
    \item every function is a container
    \item can be invoked by a web request
    \item can also call other functions
  \end{itemize}
\end{frame}

\begin{frame}{Functions -- Architecture}
  \begin{itemize}
    \item \texttt{database} - interacting with the MongoDB database
    \item \texttt{register-device} - register new devices in database
    \item \texttt{log-data} - log sensor data in database
    \item \texttt{devices} - get list of registered devices
    \item \texttt{filter} - get filtered sensor data
    \item \texttt{ui} - show web interface
  \end{itemize}
\end{frame}

\begin{frame}{Functions -- OpenFaaS Watchdog}
  \begin{itemize}
    \item needed by every function
    \item interface between OpenFaaS Gateway and function code
    \item multiple modes:
      \begin{itemize}
        \item HTTP -- forwards requests to function process running web server
        \item Serializing -- forks funcion process and passes \texttt{stdin}/\texttt{stdout} (legacy)
        \item Static -- used for serving static files
      \end{itemize}
  \end{itemize}
\end{frame}

\begin{frame}{Functions -- Building a Function Template}
  \begin{itemize}
    \item \texttt{Dockerfile}
      \begin{itemize}
        \item expects \texttt{function} directory containing source code
        \item installs all dependencies
      \end{itemize}
    \item \texttt{template.yml}
      \begin{itemize}
        \item contains metadata like the programming language or build options
      \end{itemize}
    \item common code
      \begin{itemize}
        \item web server -- avoid duplication in function source code
        \item helper functions -- i.e. for calling other functions
      \end{itemize}
  \end{itemize}
\end{frame}

\begin{frame}{Functions -- Rust Function Template}
  \begin{itemize}
    \item watchdog uses \texttt{http} mode
    \item function handler implemented with \texttt{hyper} crate (web client/server)
    \item handler forwards request to function handler
    \item function source code only has to contain the function handler
    \item helper module
      \begin{itemize}
        \item accessible from function handler
        \item reuse \texttt{hyper} dependency for calling other functions
        \item helper function for getting secrets (e.g. database password)
      \end{itemize}
  \end{itemize}
\end{frame}

\begin{frame}{Functions -- The \texttt{database} Function}
  \begin{itemize}
    \item core building block
    \item \texttt{mongodb-rust-driver} crate for interacting with MongoDB
    \item \texttt{lazy\_static} crate for persisting connection across function calls
    \item supported actions:
      \begin{itemize}
        \item \texttt{insert}
        \item \texttt{insert\_or\_update}
        \item \texttt{find}
        \item \texttt{aggregate}
        \item \texttt{update}
      \end{itemize}
    \item actions callable from other functions
  \end{itemize}
\end{frame}

\begin{frame}{Functions -- The \texttt{register-device} Function}
  \begin{itemize}
    \item called before a device starts sending data
    \item triggered when a message is posted to the \texttt{register-device} Kafka topic
    \item incoming request must contain
      \begin{itemize}
        \item a device ID
        \item a device name
      \end{itemize}
    \item calls \texttt{database} function with \texttt{insert\_or\_update} action
  \end{itemize}
\end{frame}

\begin{frame}{Functions -- The \texttt{log-data} Function}
  \begin{itemize}
    \item called when a device posts new data
    \item triggered when a message is posted to the \texttt{log-data} Kafka topic
    \item incoming request must contain
      \begin{itemize}
        \item a device ID
        \item a valid timestamp
        \item a supported data type
      \end{itemize}
    \item data type is added to the list of supported types for the device
    \item data is persisted in the corresponding collection
  \end{itemize}
\end{frame}

\begin{frame}{Functions -- The \texttt{devices} Function}
  \begin{itemize}
    \item called from the Serverless UI when showing device list
    \item calls the \texttt{database} function with the \texttt{find} action
    \item transforms the retrieved MongoDB documents to JSON
    \item returns an array of devices in JSON format
  \end{itemize}
\end{frame}

\begin{frame}{Functions -- The \texttt{filter} Function}
  \begin{itemize}
    \item called from the Serverless UI when showing data for a device
    \item calls the \texttt{database} function with either
      \begin{itemize}
        \item \texttt{find} (incoming request without time frame)
        \item \texttt{aggregate} (incoming request with time frame)
      \end{itemize}
    \item maps the MongoDB documents to JSON
    \item returns array of values for the data type specified in the incoming request
  \end{itemize}
\end{frame}

\begin{frame}{Functions -- The \texttt{ui} Function}
  \begin{itemize}
    \item not written in Rust
    \item watchdog uses \texttt{static} mode
    \item function template
    \begin{itemize}
      \item compiles JavaScript application into single bundle using Webpack
      \item final image only contains watchdog, \texttt{index.html} and \texttt{bundle.js}
    \end{itemize}
  \end{itemize}
\end{frame}
